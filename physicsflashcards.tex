%
%    physicsflashcards
%   =============
%    Copyright (C) 2013  Magdalen Berns (with some flashcard formatting help from users of tex.stackexchange.com)
%    
%   This program is free software: you can redistribute it and/or modify
%    it under the terms of the GNU General Public License as published by
%   the Free Software Foundation, either version 3 of the License, or
%    (at your option) any later version.
%
%    This program is distributed in the hope that it will be useful,
%    but WITHOUT ANY WARRANTY; without even the implied warranty of
%    MERCHANTABILITY or FITNESS FOR A PARTICULAR PURPOSE.  See the
%    GNU General Public License for more details.
%
%    You should have received a copy of the GNU General Public License
%    along with this program.  If not, see <http://www.gnu.org/licenses/>.
%

\documentclass{article}
\usepackage[paperwidth=.47\paperwidth,paperheight=.25\paperheight]{geometry}
\usepackage{pgfpages}
\usepackage[english]{babel}
\usepackage{amsfonts}
\usepackage{amsmath}
\pagestyle{empty}
\thispagestyle{empty}

\pgfpagesuselayout{8 on 1}[a4paper]        
\makeatletter
\@tempcnta=1\relax
\loop\ifnum\@tempcnta<9\relax
\pgf@pset{\the\@tempcnta}{bordercode}{\pgfusepath{stroke}}

\advance\@tempcnta by 1\relax
\repeat
\makeatother


\newenvironment{flashcard}[2][]
  {\noindent\textsc{\Large#1}\par\vfill
   {\centering\Large#2\par}
   \vfill
   \newpage\Large\centering
  }
  {\newpage}
\begin{document}
%%%%%%%%%%%%%%%%%%%%%%%%%%%%%%%%%%%%%%

\begin{flashcard}[Thermodynamics]{Describe what is meant by a system}

\vspace*{\stretch{1}}
A system is a particular part of the universe.
\vspace*{\stretch{1}}
\newline

\end{flashcard}
%%%%%%%%%%%%%%%%%%%%%%%%%%%%%%%%%%%%%%


\begin{flashcard}[Thermodynamics]{Describe what is meant by the surroundings of a system}

\vspace*{\stretch{1}}
The part of the universe which is outside (i.e. surrounding) a system.
\vspace*{\stretch{1}}
\end{flashcard}

%%%%%%%%%%%%%%%%%%%%%%%%%%%%%%%%%%%%%%


\begin{flashcard}[Thermodynamics]{Describe the boundary of a system}

\vspace*{\stretch{1}}
The boundary (or wall) of a system is the thing which separates it from its surroundings. 

\vspace*{\stretch{1}}
\end{flashcard}


%%%%%%%%%%%%%%%%%%%%%%%%%%%%%%%%%%%%%%

\begin{flashcard}[Electromagnetism]{Explain Gauss' law for an electric field in words.}

\vspace*{\stretch{1}}
The total electric flux through any closed surface is proportional to its enclosed charge.
\vspace*{\stretch{1}}
\end{flashcard}

%%%%%%%%%%%%%%%%%%%%%%%%%%%%%%%%%%%%%%

\begin{flashcard}[Thermodynamics]{Describe a closed system}

\vspace*{\stretch{1}}
A closed system is a system where no matter is exchanged, only energy.
\vspace*{\stretch{1}}
\end{flashcard}

%%%%%%%%%%%%%%%%%%%%%%%%%%%%%%%%%%%%%%

\begin{flashcard}[Thermodynamics]{Describe how adiabatic walls.}

\vspace*{\stretch{1}}
Adiabatic walls prevent thermal interaction (i.e. heat exchange) 
\vspace*{\stretch{1}}
\end{flashcard}

%%%%%%%%%%%%%%%%%%%%%%%%%%%%%%%%%%%%%%

\begin{flashcard}[Thermodynamics]{What type of walls does a thermally isolated system have?}
\vspace*{\stretch{1}}
A thermally isolated system has adiabatic walls.
\vspace*{\stretch{1}}
\end{flashcard}

%%%%%%%%%%%%%%%%%%%%%%%%%%%%%%%%%%%%%%

\begin{flashcard}[Electromagnetism]{What is Maxwell's II  and what does it express?}
\vspace*{\stretch{1}}
$$\nabla \cdot \mathbf{B} = 0$$
There are no magnetic monopoles.
\vspace*{\stretch{1}}
\end{flashcard}

%%%%%%%%%%%%%%%%%%%%%%%%%%%%%%%%%%%%%%

\begin{flashcard}[Statistical Mechanics]{What is the equilibrium entropy of an isolated system of N constituents with energy E?}
Equilibrium entropy of an isolated system is expressed as follows:
\vspace*{\stretch{1}}
$$S(N, E) = \mathrm{k ln \Omega(N, E, {\alpha*})}$$
\vspace*{\stretch{1}}

\end{flashcard}

%%%%%%%%%%%%%%%%%%%%%%%%%%%%%%%%%%%%%%

\begin{flashcard}[Diffraction physics]{Define the electric displacement of a dielectric.}
\vspace*{\stretch{1}}
$$\underline{D}=\epsilon_0 \underline{E}+\underline{P}$$
Where $ \mathbf{E}$ is the electric field, $\mathbf{P}$ is the polarisation
\vspace*{\stretch{1}}

\end{flashcard}

%%%%%%%%%%%%%%%%%%%%%%%%%%%%%%%%%%%%%%

\begin{flashcard}[Diffraction Physics]{What are the refractive indices of $\mathrm{Mg F_{2}}$ and $\mathrm{Al_{2}O_{3}}$ ?}

\begin{table}[htdp]
\caption{A table of refractive indices}
\begin{tabular}{| r | c | r | c | r |}
\hline
Material & Symbol & n  \\
\hline
Magnesium Fluoride & $\mathrm{Mg F_{2}}$ & 1.38 \\
Aluminium Oxide & $\mathrm{Al_{2}O_{3}}$ & 1.62 \\
\hline
\end{tabular}
\label{default}
\end{table}%
\end{flashcard}
%%%%%%%%%%%%%%%%%%%%%%%%%%%%%%%%%%%%%%
\begin{flashcard}[Quantum Mechanics]{What is the wave function for a free particle?}
The quantum mechanical interpretation of a free particle is expressed as follows
\vspace*{\stretch{1}}
$$\psi(\mathbf{r},t)=C e^{\frac{i}{\hbar}(\mathbf{p}\cdot \mathbf{r}-\epsilon t)}$$
\vspace*{\stretch{1}}

\end{flashcard}

%%%%%%%%%%%%%%%%%%%%%%%%%%%%%%%%%%%%%%

\begin{flashcard}[Electromagnetism]{Express the force between two charges.}
The force between two charges, $q_{1}$ and $q_{2}$ that a separated by a distance $r$ should be expressed as follows:
\vspace*{\stretch{1}}
$$\mathbf{F}=\frac{q_{1} q_{2}}{4\pi \epsilon_{0} r^2} \mathbf{r} $$
\vspace*{\stretch{1}}
Where $\epsilon_{0}$ is the permittivity of free space.

\end{flashcard}

%%%%%%%%%%%%%%%%%%%%%%%%%%%%%%%%%%%%%%


\begin{flashcard}[Electromagnetism]{What is the total charge contained within a volume?}
\vspace*{\stretch{1}}
$$Q_v=\int_{V}{\rho(\underline{r})dV}$$
Where $\rho$ is the sum of the charges.\vspace*{\stretch{1}}

\end{flashcard}

%%%%%%%%%%%%%%%%%%%%%%%%%%%%%%%%%%%%%%


\begin{flashcard}[Diffraction Physics]{What happens when we apply a field to an electrically insulating material such as glass?}
\vspace*{\stretch{1}}
The charges will separate causing a polarisation in the material.
\vspace*{\stretch{1}}

\end{flashcard}

%%%%%%%%%%%%%%%%%%%%%%%%%%%%%%%%%%%%%%








\end{document}