%
%    physicsflashcards
%   =============
%    Copyright (C) 2013  Magdalen Berns (with some flashcard formatting help from users of tex.stackexchange.com)
%    
%   This program is free software: you can redistribute it and/or modify
%    it under the terms of the GNU General Public License as published by
%   the Free Software Foundation, either version 3 of the License, or
%    (at your option) any later version.
%
%    This program is distributed in the hope that it will be useful,
%    but WITHOUT ANY WARRANTY; without even the implied warranty of
%    MERCHANTABILITY or FITNESS FOR A PARTICULAR PURPOSE.  See the
%    GNU General Public License for more details.
%
%    You should have received a copy of the GNU General Public License
%    along with this program.  If not, see <http://www.gnu.org/licenses/>.
%

\documentclass{article}
\usepackage[paperwidth=.5\paperwidth,paperheight=.25\paperheight]{geometry}
\usepackage{pgfpages}
\usepackage[english]{babel}
\usepackage[T1]{fontenc}
\usepackage[latin1]{inputenc}
\usepackage{color}
\usepackage{float}
\usepackage{parskip}
\usepackage{amsfonts}
\usepackage{amsmath}
\pagestyle{empty}
\usepackage{caption}
\thispagestyle{empty}
\pgfpagesuselayout{8 on 1}[a4paper]

%\mode<presentation>
%{
%    \usetheme{Warsaw}
%    \useoutertheme[subsection=false]{miniframes}
%    \usefonttheme[onlylarge]{structuresmallcapsserif}
%    \setbeamercovered{transparent}
%    \setbeamercolor{title}{fg=red!80!black}
%}
%
%\pagestyle{empty}
%\thispagestyle{empty}
%\pgfpagesuselayout{8 on 1}[a4paper,
%        , border shrink=2mm
%        ]
%        
        


\makeatletter
\@tempcnta=1\relax
\loop\ifnum\@tempcnta<9\relax
\pgf@pset{\the\@tempcnta}{bordercode}{\pgfusepath{stroke}}

\advance\@tempcnta by 1\relax
\repeat
\makeatother


\newenvironment{flashcard}[2][]{%
\noindent  \textsc{#1}

\vfill
\centerline{{\Large{#2}}}
\vfill
\newpage
}
{\newpage}



\begin{document}
%%%%%%%%%%%%%%%%%%%%%%%%%%%%%%%%%%%%%%

\begin{flashcard}[Thermodynamics]{Describe a system}
\centering
{Describe a system}

\vspace*{\stretch{1}}
A system is a particular part of the universe.
\vspace*{\stretch{1}}
\newline

\end{flashcard}
%%%%%%%%%%%%%%%%%%%%%%%%%%%%%%%%%%%%%%


\begin{flashcard}[Thermodynamics]{Describe the surroundings of a system}
\centering
{Describe the surroundings of a system}

\vspace*{\stretch{1}}
The part of the universe which is outside (i.e. surrounding) a system.
\vspace*{\stretch{1}}
\end{flashcard}

%%%%%%%%%%%%%%%%%%%%%%%%%%%%%%%%%%%%%%


\begin{flashcard}[Thermodynamics]{Describe the boundary of a system}
Describe the boundary of a system.

\vspace*{\stretch{1}}
The boundary (or wall) of a system is the thing which separates it from its surroundings. 

\vspace*{\stretch{1}}
\end{flashcard}


%%%%%%%%%%%%%%%%%%%%%%%%%%%%%%%%%%%%%%

\begin{flashcard}[Thermodynamics]{Describe the boundary of a system}
Describe the boundary of a system.

\vspace*{\stretch{1}}
The boundary (or wall) of a system is the thing which separates it from its surroundings. 
\vspace*{\stretch{1}}
\end{flashcard}

%%%%%%%%%%%%%%%%%%%%%%%%%%%%%%%%%%%%%%

\begin{flashcard}[Thermodynamics]{Describe a closed system}
Describe a closed system.

\vspace*{\stretch{1}}
A closed system is a system where no matter is exchanged, only energy.
\vspace*{\stretch{1}}
\end{flashcard}

%%%%%%%%%%%%%%%%%%%%%%%%%%%%%%%%%%%%%%

\begin{flashcard}[Thermodynamics]{Describe how adiabatic walls.}

Describe adiabatic walls.

\vspace*{\stretch{1}}
Adiabatic walls prevent thermal interaction (i.e. heat exchange) 
\vspace*{\stretch{1}}
\end{flashcard}

%%%%%%%%%%%%%%%%%%%%%%%%%%%%%%%%%%%%%%

\begin{flashcard}[Thermodynamics]{What type of walls does a thermally isolated system have?}

What type of walls does a thermally
 isolated system have?

\vspace*{\stretch{1}}
A thermally isolated system has adiabatic walls.
\vspace*{\stretch{1}}
\end{flashcard}

%%%%%%%%%%%%%%%%%%%%%%%%%%%%%%%%%%%%%%


\begin{flashcard}[Electromagnetism]{What is Maxwell's II and what does it tell us?}

What is Maxwell's II and what does it tell us?

\vspace*{\stretch{1}}
$$\nabla \cdot \mathbf{B} = 0$$
\vspace*{\stretch{1}}
\end{flashcard}


\end{document}